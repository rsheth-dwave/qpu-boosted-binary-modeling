\section{Experiments}

Utilizing \cref{eq:ae-training-objective} on FFHQ \TODO{verify} with a $3 \times 256 \times 256 \rightarrow 4 \times 16 \times 16$ encoding and then sampling latent codes from the learned QPU$( ;J^\star, h^\star, \theta_\text{ann} )$ yields the results in \Cref{fig:qpu-sampled-image}.
\begin{figure}
    \centering
    \includegraphics[width=0.9\linewidth]{figs/qpu-sampled-image.png}
    \caption{Generated samples from the learned QPU.}%
    \label{fig:qpu-sampled-image}
\end{figure}
Performing discrete flow matching to learn the ``velocity field'' between 
${\{ z_i \}}_{i=1}^{N}$ and (generated) samples from the learned QPU yields the results in \Cref{fig:flow-matched-qpu-coupled-sampled-image}.
\begin{figure}
    \centering
    \includegraphics[width=0.9\linewidth]{figs/flow-matched-qpu-coupled-sampled-image.png}
    \caption{Generated samples after flow matching to the learned QPU.}%
    \label{fig:flow-matched-qpu-coupled-sampled-image}
\end{figure}
\NOTE{A 4-bit latent code represents a very small codebook.\ \citet{wang2023binarylatentdiffusion} utilized 32-, 16-, and 8-bit codebooks in their experiments.}
\TODO{Add diversity metric(s) to evaluation, e.g.,~\cite{kynkäänniemi2019improvedprecisionrecallmetric} / \url{https://github.com/kynkaat/improved-precision-and-recall-metric} (?)} % chktex 36
